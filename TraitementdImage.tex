% Created 2019-02-07 jeu. 14:39
% Intended LaTeX compiler: pdflatex
\documentclass[11pt]{article}
\usepackage[utf8]{inputenc}
\usepackage[T1]{fontenc}
\usepackage{graphicx}
\usepackage{grffile}
\usepackage{longtable}
\usepackage{wrapfig}
\usepackage{rotating}
\usepackage[normalem]{ulem}
\usepackage{amsmath}
\usepackage{textcomp}
\usepackage{amssymb}
\usepackage{capt-of}
\usepackage{hyperref}
\usepackage{minted}
\usepackage[french]{babel}
\usepackage[x11names]{xcolor}
\hypersetup{linktoc = all, colorlinks = true, urlcolor = DodgerBlue4, citecolor = PaleGreen1, linkcolor = black}
\author{Raoul HATTERER}
\date{\today}
\title{Traitement d'image\\\medskip
\large (1 séance = 1h30)}
\hypersetup{
 pdfauthor={Raoul HATTERER},
 pdftitle={Traitement d'image},
 pdfkeywords={},
 pdfsubject={},
 pdfcreator={Emacs 26.1 (Org mode 9.2)}, 
 pdflang={French}}
\begin{document}

\maketitle
\tableofcontents

Source : \href{http://www.ac-grenoble.fr/disciplines/informatiquelycee/n\_site/snt\_photo\_transImg.html}{Traitement d'image de l'académie de grenoble}

\section{Installation de PIL}
\label{sec:orgb2ae7e2}

À faire au préalable par le professeur.

\begin{minted}[]{shell}
 pip3 install pillow
\end{minted}


\section{Image de départ}
\label{sec:org840a73b}

\begin{figure}[htbp]
\centering
\includegraphics[width=.9\linewidth]{pomme.jpg}
\caption{Image de départ (480px \texttimes{} 300px)}
\end{figure}


\section{Comment lire un pixel}
\label{sec:org12a52f7}

Après avoir fait quelques recherches sur ce qu'est un "pixel", voyons comment lire le pixel de coordonnées (100,250).

\begin{minted}[]{python}
from PIL import Image
img = Image.open("pomme.jpg")
r,v,b=img.getpixel((100,250))
print("canal rouge : ",r,"canal vert : ",v,"canal bleu : ",b)
\end{minted}

\begin{verbatim}
('canal rouge : ', 19, 'canal vert : ', 88, 'canal bleu : ', 192)
\end{verbatim}


\section{Comment écrire un pixel}
\label{sec:org006845f}

\begin{minted}[]{python}
from PIL import Image
img = Image.open("pomme.jpg")
img.putpixel((5,5),(255,0,0))
img.show()
\end{minted}


\section{Que fait le programme suivant ?}
\label{sec:org06eb80c}

\begin{minted}[]{python}
from PIL import Image
img = Image.open("pomme.jpg")
largeur_image=480
hauteur_image=300
for y in range(hauteur_image):
    for x in range(largeur_image):
        rouge,vert,bleu=img.getpixel((x,y))
        nouveau_rouge=vert
        nouveau_vert=bleu
        nouveau_bleu=rouge
        img.putpixel((x,y),(nouveau_rouge,nouveau_vert,nouveau_bleu))
img.show()
img.save("pommeMystere.jpg")
\end{minted}


On analyse le code ci-dessus qui servira de base pour le défi suivant.

\begin{figure}[htbp]
\centering
\includegraphics[width=.9\linewidth]{pommeMystere.jpg}
\caption{Résultat du programme mystère}
\end{figure}


\section{Passage d'une image en niveau de gris}
\label{sec:orgfd80acc}

Après avoir fait quelques recherches sur les "images en niveau de gris", écrivez un programme qui transforme une "image couleur" en une "image en niveau de gris".

Petite astuce qui pourrait vous aider : en Python pour avoir une division entière (le résultat est un entier), il faut utiliser l'opérateur // à la place de l'opérateur / 

Remarque: On donne l'algorithme aux élèves (ou on le construit avec eux) ; ils doivent alors programmer le passage d'une image couleur à une image en niveaux de gris.


\begin{minted}[]{python}
from PIL import Image
img = Image.open("pomme.jpg")
largeur_image=480
hauteur_image=300
for y in range(hauteur_image):
    for x in range(largeur_image):
       rouge,vert,bleu=img.getpixel((x,y))
       nouveau_rouge=(vert+bleu+rouge)//3
       nouveau_vert=(vert+bleu+rouge)//3
       nouveau_bleu=(vert+bleu+rouge)//3
       img.putpixel((x,y),(nouveau_rouge,nouveau_vert,nouveau_bleu))
img.show()
img.save("pommegrise.jpg")
\end{minted}

\begin{figure}[htbp]
\centering
\includegraphics[width=.9\linewidth]{pommegrise.jpg}
\caption{Image en niveaux de gris}
\end{figure}
\end{document}